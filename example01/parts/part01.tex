\section{Цели работы}

\begin{enumerate}
    \item Научиться создавать настольные приложения с графическим интерфейсом \guillemotleft Java FX\guillemotright~для платформы \guillemotleft Java\guillemotright~
    (далее --- \guillemotleft приложения\guillemotright) в интегрированной среде разработки (далее --- \guillemotleft IDE\guillemotright);
    \item Эксплуатировать IDE в режиме разработки и отладки;
    \item Собирать приложения для режима эксплуатации.
\end{enumerate}

\section{Ход работы}

Для достижения поставленных целей необходимо последовательно выполнить работы по следующему порядку:

\subsection{Создание проекта в IDE}

Для создания проекта в IDE \guillemotleft IntelliJ IDEA CE\guillemotright необходимо нажать на кнопку \guillemotleft создать \guillemotright, как показано на изображении \ref{ris:00_01}

\begin{figure}[h]
    \centering
    \begin{tikzpicture}
        \node[draw=gray, line width=1pt, inner sep=0pt] % Рамка с толщиной линии 1pt
        {\includegraphics[width=10cm]{images/01}}; % Здесь можно регулировать ширину картинки в миллиметрах
    \end{tikzpicture}
    \caption{Создание приложения. Шаг 01.} % Подпись к картинке
    \label{ris:00_01}
\end{figure}

\subsection{Запуск приложения в режиме разработки}

Пример кода

\begin{verbatim}
    // Пример кода моноширинным текстом
    public class HelloApplication extends Application {
    @Override
    public void start(Stage stage) throws IOException {
        FXMLLoader fxmlLoader =
    new FXMLLoader(HelloApplication.class
    .getResource("hello-view.fxml"));
        Scene scene = new Scene(fxmlLoader.load(), 320, 240);
        stage.setTitle("Hello!");
        stage.setScene(scene);
        stage.show();
    }

    public static void main(String[] args) {
        launch();
    }
}
\end{verbatim}

\subsection{Запуск приложения в режиме отладки}

И так далее...

\subsection{Сборка приложения для режима эксплуатации}

И тому подобное...
